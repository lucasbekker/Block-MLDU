\chapter{Introduction}

The goal of this Capita Selecta is to investigate the possibilities for a fast MATLAB based implementation of the Block MLDU algorithm.\\

\noindent High performance solvers are usually written in languages like C/C++ because they are compiled, avoiding the sometimes costly step of the interpreter. Languages like MATLAB are dynamically typed and interpreter based, which make them great for prototyping purposes.\\

\noindent The Block MLDU algorithm shows great possibilities, but the lack of a high performance implementation makes it prohibitive to use for large matrices. This is a problem because research projects sometimes encounter very large matrices that need to be solved with the Block MLDU algorithm.\\

\noindent This project tries to alleviate the problem by investigating the performance bottlenecks of a MATLAB based implementation and provide workarounds for them. This should result in a much faster implementation without requiring the extra work associated with a full C/C++ version.\\

\noindent The biggest advantage of this workflow is that the time required to implement a workaround for a bottleneck will be much lower in MATLAB than in C/C++, making it easier to try different approaches to the problems. The lessons learned at this stage will still be useful when the code does finally get ported to C/C++.\\

\noindent The biggest disadvantage to sticking to MATLAB is that the interpreter will always provide some overhead compared to C/C++. The hope is that this overhead will be small compared to the other bottlenecks encountered.

\newpage

\section{The Block MLDU algorithm}

The Block MLDU algorithm is very comparable to the well known LU decomposition. The main difference between the Block MLDU algorithm and LU decomposition is the "block" nature of the splitting.

\subsection{Example}

A simple example will be provided to highlight the differences between LU and Block MLDU. Let matrix $A$ be $4 \times 4$:\\

$
\hspace{20mm} A = 
\left\lbrack\begin{array}{rrrr}
           1&           0&           1&           0\\
           2&           1&          -1&           0\\
           0&          -1&           0&           1\\
          -2&           0&           1&           3
\end{array}\right\rbrack
$
$
\hspace{30mm} s =
\left\lbrack\begin{array}{rrr}
           2&           1&           1
\end{array}\right\rbrack
$\\

\noindent As stated by row $s$, matrix $A$ will first be split using a $2 \times 2$ block, followed by two $1 \times 1$ blocks.\\

\noindent \textbf{Step one:}\\

$
\left\lbrack\begin{array}{rr|rr}
          D1&          D1&          U1&          U1\\
          D1&          D1&          U1&          U1\\ \hline
          L1&          L1&          M1&          M1\\
          L1&          L1&          M1&          M1
\end{array}\right\rbrack
$
$
\hspace{10mm} L1 = 
\left\lbrack\begin{array}{rr}
           0&          -1\\
          -2&           0
\end{array}\right\rbrack
$
$
\hspace{6mm} D1 = 
\left\lbrack\begin{array}{rr}
           1&           0\\
           2&           1
\end{array}\right\rbrack
$
$
\hspace{6mm} U1 = 
\left\lbrack\begin{array}{rr}
           1&           0\\
          -1&           0
\end{array}\right\rbrack
$\\

$
Schur1 = L1 D1^{-1} U1 =
$
$
\left\lbrack\begin{array}{rr}
           3&           0\\
          -2&           0
\end{array}\right\rbrack
$
$
\hspace{20mm} A - Schur1 = 
\left\lbrack\begin{array}{rrrr}
           1&           0&           1&           0\\
           2&           1&          -1&           0\\
           0&          -1&  \textbf{-3}&  \textbf{1}\\
          -2&           0&   \textbf{3}&  \textbf{3}
\end{array}\right\rbrack
$\\

\noindent \textbf{Step two:}\\

$
\left\lbrack\begin{array}{rr|r|r}
          D1&          D1&          U1&          U1\\
          D1&          D1&          U1&          U1\\ \hline
          L1&          L1&          D2&          U2\\ \hline
          L1&          L1&          L2&          M2
\end{array}\right\rbrack
$
$
\hspace{10mm} L2 = 
\left\lbrack\begin{array}{r}
           3
\end{array}\right\rbrack
$
$
\hspace{18mm} D2 = 
\left\lbrack\begin{array}{r}
           -3
\end{array}\right\rbrack
$
$
\hspace{10mm} U2 = 
\left\lbrack\begin{array}{r}
           1
\end{array}\right\rbrack
$\\

$
Schur2 = L2 D2^{-1} U2 =
$
$
\left\lbrack\begin{array}{r}
           -1
\end{array}\right\rbrack
$
$
\hspace{12mm} A - Schur1 - Schur2 = 
\left\lbrack\begin{array}{rrrr}
           1&           0&           1&           0\\
           2&           1&          -1&           0\\
           0&          -1&          -3&           1\\
          -2&           0&           3&   \textbf{4}
\end{array}\right\rbrack
$\\

\noindent \textbf{Step three:}\\

$
\left\lbrack\begin{array}{rr|r|r}
          D1&          D1&          U1&          U1\\
          D1&          D1&          U1&          U1\\ \hline
          L1&          L1&          D2&          U2\\ \hline
          L1&          L1&          L2&          D3
\end{array}\right\rbrack
\\$\\

\noindent \textbf{Result:}\\

$
L = 
\left\lbrack\begin{array}{rrrr}
           0&           0&           0&           0\\
           0&           0&           0&           0\\
   \textbf{0}& \textbf{-1}&          0&           0\\
  \textbf{-2}&  \textbf{0}&  \textbf{3}&          0
\end{array}\right\rbrack
$
$
\hspace{10mm} D =
\left\lbrack\begin{array}{rrrr}
   \textbf{1}&  \textbf{0}&          0&           0\\
   \textbf{2}&  \textbf{1}&          0&           0\\
           0&           0&  \textbf{-3}&          0\\
           0&           0&           0&   \textbf{4}
\end{array}\right\rbrack
$
$
\hspace{10mm} U =
\left\lbrack\begin{array}{rrrr}
           0&           0&   \textbf{1}&  \textbf{0}\\
           0&           0&  \textbf{-1}&  \textbf{0}\\
           0&           0&           0&   \textbf{1}\\
           0&           0&           0&           0
\end{array}\right\rbrack
\\$\\

$
\\
A = (L + D)D^{-1}(D + U)
$
